%seed - 52766329910

\documentclass[12pt]{article}

\usepackage[margin=0.5in]{geometry}
\usepackage{amsmath}
\usepackage{amsfonts}
\usepackage{amssymb}
\usepackage{graphicx}
\usepackage{tikz}
\usepackage{tcolorbox}
\usepackage[shortlabels]{enumitem}
\usepackage{ifthen}
\usepackage{xcolor}
\usepackage{tasks}

% change this value to produce answer keys for exams
\newboolean{make_key}
\setboolean{make_key}{true}
\newcommand{\version}{}
% declarations for the different parts of the exam
\newtcolorbox{instructionbox}{
	colback = gray!25!white, 
	colframe = black!50!white, 
	boxrule = 0.4pt, 
	arc = 0pt
}

\newcommand{\iskey}[1]{\ifthenelse{\boolean{make_key}}{{\color{red}#1}}{}}

\begin{document}

% remove default page numbers
\pagestyle{empty}

% show the title information and student name
\noindent Not a real exam Version {\version} \hfill Name: \rule{6cm}{0.15mm} \vspace{2mm}

% show the instructions for the exam
\begin{instructionbox}
    \textbf{Instructions:} This is just a test of the emergency exam broadcast system.
    If it were a real exam, this note would be followed by instructions on how to complete
    the exam and indications about what kinds of resources are available during said exam. 
\end{instructionbox}

\raggedright

\begin{enumerate}[1.]


\item (2 pts) 
Solve the polynomial equation: $ x^{2} - 3 x - 10 = 0 $. \\[4mm]
\iskey{
    \begin{tabular}{ccl}
        $x^{2} - 3 x - 10$ & $=$ & $(x + 2)(x - 5)$ \\[2mm]
        & $\Rightarrow$ & $x \in \left\{ \mathtt{\text{-2, 5}} \right\}$ \\
    \end{tabular}
}
\item (2 pts) 
Solve the polynomial equation: $ x^{2} + 3 x = 0 $. \\[4mm]
\iskey{
    \begin{tabular}{ccl}
        $x^{2} + 3 x$ & $=$ & $(x + 3)(x)$ \\[2mm]
        & $\Rightarrow$ & $x \in \left\{ \mathtt{\text{-3, 0}} \right\}$ \\
    \end{tabular}
}


\vspace{3cm}


\item (4 pts) Solve the polynomial equation: $ 4 x^{2} - 2 x - 14 = 0 $. \\[4mm]

\iskey{
    $\displaystyle 4 x^{2} - 2 x = 14 $ \\[4mm]
    $\displaystyle x^{2} - \frac{x}{2} = \frac{7}{2} $ \\[4mm]
    $\displaystyle x^{2} - \frac{x}{2} + \frac{1}{16} = \frac{57}{16} $ \\[4mm]
    $\displaystyle \left(x - \frac{1}{4}\right)^{2} = \frac{57}{16} $ \\[4mm]
    $\displaystyle \left(x - \frac{1}{4}\right)^{2} = \frac{57}{16} $ \\[4mm]
    $\displaystyle x - \frac{1}{4} = \pm \frac{\sqrt{57}}{4} $ \\[4mm]

    $\displaystyle x \in \left\{ \frac{1}{4} - \frac{\sqrt{57}}{4}, \frac{1}{4} + \frac{\sqrt{57}}{4} \right\}$
}

\end{enumerate}
\end{document}