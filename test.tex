\documentclass{article}

\usepackage[margin=0.5in]{geometry}
\usepackage{amsmath}
\usepackage{amsfonts}
\usepackage{amssymb}
\usepackage{graphicx}
\usepackage{tikz}
\usepackage{tcolorbox}
\usepackage[shortlabels]{enumitem}
\usepackage{ifthen}
\usepackage{xcolor}
\usepackage{tasks}

% change this value to produce answer keys for exams
\newboolean{make_key}
\setboolean{make_key}{true}

% declarations for the different parts of the exam
\newtcolorbox{instructionbox}{
	colback = gray!25!white, 
	colframe = black!50!white, 
	boxrule = 0.4pt, 
	arc = 0pt
}

\providecommand{\tightlist}{\setlength{\itemsep}{0pt}\setlength{\parskip}{0pt}}
\newcommand{\newquestion}{\vspace{4mm} \noindent}
\newcommand{\iskey}[1]{\ifthenelse{\boolean{make_key}}{{\color{red}#1}}{}}

\begin{document}

% remove default page numbers
\pagestyle{empty}

% show the title information and student name
\noindent Not a real exam \hfill Name: \rule{6cm}{0.15mm} \vspace{2mm}

% show the instructions for the exam
\begin{instructionbox}
    \textbf{Instructions:} You are permitted the use of a formula sheet that you 
    create, which may contain formulas or hints as to where to find items in your 
    calculator, but no worked out examples. You may use any calculator you like, 
    including the online Stats Calculator, but may not use any apps on your phone 
    for computations.
\end{instructionbox}

\raggedright

\newquestion
%python
%data = rands(18, 55, 8)
%table = showdataarray(data)
%avg = mean(data)
%med = median(data)
%mode = modes(data)
%std = stdev(data)
%end
\item (6 pts) The highway mileage (mpg) for a sample of 8 different models
of a car company can be found below. Find the mean, median, mode, and
standard deviation. Round to one decimal place as needed. \\[2mm]

@table@

\vspace{2mm}

\begin{tasks}[label={}](2)
  \task Mean = \rule{4cm}{0.15mm}
  \task Range = \rule{4cm}{0.15mm}  \\[4mm]
  \task Median = \rule{4cm}{0.15mm}
  \task Standard Deviation = \rule{4cm}{0.15mm} \\[4mm]
  \task Mode = \rule{4cm}{0.15mm}
\end{tasks}

\newquestion
%python
%z0 = rand(-5, -1)
%z1 = rand(z0, 6)
%c = z0 * z1
%b = z0 + z1
%x = symbols('x')
%fact0 = x - z0
%fact1 = x - z1
%poly = expand(fact0 * fact1)
%print(latex(poly))
%end
2. (4 pts) Factor the polynomial $ @latex(poly)@ $.
\iskey{$@latex(poly)@ = @f'({latex(fact0)})({latex(fact1)})'@$}

\newquestion
2. (4 pts) Given the distribution below, find the probability of the 
shaded region. \vspace{4mm}
%\begin{center}
%	\includegraphics[width=4.125in,height=1.79167in]{media/image1.png}
%\end{center}


















\end{document}