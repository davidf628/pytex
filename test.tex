\documentclass[12pt]{article}

\usepackage[margin=0.5in]{geometry}
\usepackage{amsmath}
\usepackage{amsfonts}
\usepackage{amssymb}
\usepackage{graphicx}
\usepackage{tikz}
\usepackage{tcolorbox}
\usepackage[shortlabels]{enumitem}
\usepackage{ifthen}
\usepackage{xcolor}
\usepackage{tasks}

% change this value to produce answer keys for exams
\newboolean{make_key}
\setboolean{make_key}{true}

\newcommand{\version}{C}

% declarations for the different parts of the exam
\newtcolorbox{instructionbox}{
	colback = gray!25!white, 
	colframe = black!50!white, 
	boxrule = 0.4pt, 
	arc = 0pt
}

\newcommand{\iskey}[1]{\ifthenelse{\boolean{make_key}}{{\color{red}#1}}{}}

\begin{document}

% remove default page numbers
\pagestyle{empty}

% show the title information and student name
\noindent Not a real exam Version {\version} \hfill Name: \rule{6cm}{0.15mm} \vspace{2mm}

% show the instructions for the exam
\begin{instructionbox}
    \textbf{Instructions:} This is just a test of the emergency exam broadcast system.
    If it were a real exam, this note would be followed by instructions on how to complete
    the exam and indications about what kinds of resources are available during said exam. 
\end{instructionbox}

\raggedright

\begin{enumerate}[1.]


\item (2 pts) @importrandpytexfrom('output/q1.tex', 'output/q2.tex')@

\item (2 pts) @importrandpytexfrom("output/q1.tex", "output/q2.tex")@



\vspace{3cm}

%python
%x = symbols('x')
%a = rand(1, 5)
%b = nzrand(-7, 7)
%c = nzrand(-25, 25)
%poly = a*x ** 2 + b*x + c
%sol = joinarray( solve(poly, x), func=lambda x: latex(x) )
%step1 = Eq(a*x**2 + b*x, -c)
%step2 = Eq(x**2 + Rational(b, a) * x, Rational(-c, a))
%step3 = Eq(x**2 + Rational(b, a) * x + Rational(Rational(b**2, a**2), 4), Rational(-c, a) + Rational(b**2, 4 * a**2))
%step4 = Eq((x + Rational(b, 2*a))**2, Rational(b**2-4*a*c, 4*a**2) )
%step5 = Eq( x + Rational(b, 2*a), sp.sqrt(Rational(b**2 - 4*a*c, 4*a**2)))
%end

\item (4 pts) Solve the polynomial equation: $ @latex(poly)@ = 0 $. \\[4mm]

\iskey{
    $\displaystyle @latex(step1)@ $ \\[4mm]
    $\displaystyle @latex(step2)@ $ \\[4mm]
    $\displaystyle @latex(step3)@ $ \\[4mm]
    $\displaystyle @latex(step4)@ $ \\[4mm]
    $\displaystyle @latex(step4)@ $ \\[4mm]
    $\displaystyle @latex(step5.lhs)@ = \pm @latex(step5.rhs)@ $ \\[4mm]

    $\displaystyle x \in \left\{ @sol@ \right\}$
}

\end{enumerate}
\end{document}