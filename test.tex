\documentclass[12pt]{article}

\usepackage[margin=0.5in]{geometry}
\usepackage{amsmath}
\usepackage{amsfonts}
\usepackage{amssymb}
\usepackage{graphicx}
\usepackage{tikz}
\usepackage{tcolorbox}
\usepackage[shortlabels]{enumitem}
\usepackage{ifthen}
\usepackage{xcolor}
\usepackage{tasks}
\usepackage{pgfplots}

% change this value to produce answer keys for exams
\newboolean{make_key}
\setboolean{make_key}{true}

\newcommand{\version}{C}

% declarations for the different parts of the exam
\newtcolorbox{instructionbox}{
	colback = gray!25!white, 
	colframe = black!50!white, 
	boxrule = 0.4pt, 
	arc = 0pt
}

\newcommand{\iskey}[1]{\ifthenelse{\boolean{make_key}}{{\color{red}#1}}{}}

\begin{document}

% remove default page numbers
\pagestyle{empty}

% show the title information and student name
\noindent Not a real exam Version {\version} \hfill Name: \rule{6cm}{0.15mm} \vspace{2mm}

% show the instructions for the exam
\begin{instructionbox}
    \textbf{Instructions:} This is just a test of the emergency exam broadcast system.
    If it were a real exam, this note would be followed by instructions on how to complete
\end{instructionbox}

Is this the real life?
Is this just fantasy?

\begin{enumerate}

\item Here is a histogram: \\[4mm]
  @histogram(['a', 'b', 'c'], [5, 7, 2], 8, 2)@

%python
% a = 15
% if a > 20:
%   print('it is bigger')
% else:
%   print('it is smaller')
% data = rands(13, 38, 20)
%end

\item This is a dot plot: \\[4mm]

@dotplot(data, xmin=10, xmax=40, xscl=5, xdist="3mm")@

%python
%
% # Pick random data sets
% set1 = sorta(diffrands(10, 99, 8))
%
% # ensure the first set results in a class width whose decimal is < 0.5
% range1 = max(set1) - min(set1)
% prevmax = set1[6]
% while frac(range1 / 7) > 0.5:
%   set1[7] = rand(prevmax + 1, 110)
%   range1 = max(set1) - min(set1)
%
% disp1 = joinarray(set1, ", ") 
%end

\item Find the class width of the set: @disp1@ \\[4cm]


\item (2 pts) @importpytex('q1.tex')@

@importpytex("q1.tex", "q2.tex", choose=2, prefixcode='\item (2 pts) ')@



\vspace{3cm}

%python
%x = symbols('x')
%a = rand(1, 5)
%b = nzrand(-7, 7)
%c = nzrand(-25, 25)
%poly = a*x ** 2 + b*x + c
%sol = joinarray( solve(poly, x), func=lambda x: latex(x) )
%step1 = Eq(a*x**2 + b*x, -c)
%step2 = Eq(x**2 + Rational(b, a) * x, Rational(-c, a))
%step3 = Eq(x**2 + Rational(b, a) * x + Rational(Rational(b**2, a**2), 4), Rational(-c, a) + Rational(b**2, 4 * a**2))
%step4 = Eq((x + Rational(b, 2*a))**2, Rational(b**2-4*a*c, 4*a**2) )
%step5 = Eq( x + Rational(b, 2*a), sp.sqrt(Rational(b**2 - 4*a*c, 4*a**2)))
%end

\item (4 pts) Solve the polynomial equation: $ @latex(poly)@ = 0 $. \\[4mm]

\iskey{
    $\displaystyle @latex(step1)@ $ \\[4mm]
    $\displaystyle @latex(step2)@ $ \\[4mm]
    $\displaystyle @latex(step3)@ $ \\[4mm]
    $\displaystyle @latex(step4)@ $ \\[4mm]
    $\displaystyle @latex(step4)@ $ \\[4mm]
    $\displaystyle @latex(step5.lhs)@ = \pm @latex(step5.rhs)@ $ \\[4mm]

    $\displaystyle x \in \left\{ @sol@ \right\}$
}

\end{enumerate}
\end{document}